\documentclass[11pt]{article}

    \usepackage[breakable]{tcolorbox}
    \usepackage{parskip} % Stop auto-indenting (to mimic markdown behaviour)
    
    \usepackage{iftex}
    \ifPDFTeX
    	\usepackage[T1]{fontenc}
    	\usepackage{mathpazo}
    \else
    	\usepackage{fontspec}
    \fi

    % Basic figure setup, for now with no caption control since it's done
    % automatically by Pandoc (which extracts ![](path) syntax from Markdown).
    \usepackage{graphicx}
    % Maintain compatibility with old templates. Remove in nbconvert 6.0
    \let\Oldincludegraphics\includegraphics
    % Ensure that by default, figures have no caption (until we provide a
    % proper Figure object with a Caption API and a way to capture that
    % in the conversion process - todo).
    \usepackage{caption}
    \DeclareCaptionFormat{nocaption}{}
    \captionsetup{format=nocaption,aboveskip=0pt,belowskip=0pt}

    \usepackage[Export]{adjustbox} % Used to constrain images to a maximum size
    \adjustboxset{max size={0.9\linewidth}{0.9\paperheight}}
    \usepackage{float}
    \floatplacement{figure}{H} % forces figures to be placed at the correct location
    \usepackage{xcolor} % Allow colors to be defined
    \usepackage{enumerate} % Needed for markdown enumerations to work
    \usepackage{geometry} % Used to adjust the document margins
    \usepackage{amsmath} % Equations
    \usepackage{amssymb} % Equations
    \usepackage{textcomp} % defines textquotesingle
    % Hack from http://tex.stackexchange.com/a/47451/13684:
    \AtBeginDocument{%
        \def\PYZsq{\textquotesingle}% Upright quotes in Pygmentized code
    }
    \usepackage{upquote} % Upright quotes for verbatim code
    \usepackage{eurosym} % defines \euro
    \usepackage[mathletters]{ucs} % Extended unicode (utf-8) support
    \usepackage{fancyvrb} % verbatim replacement that allows latex
    \usepackage{grffile} % extends the file name processing of package graphics 
                         % to support a larger range
    \makeatletter % fix for grffile with XeLaTeX
    \def\Gread@@xetex#1{%
      \IfFileExists{"\Gin@base".bb}%
      {\Gread@eps{\Gin@base.bb}}%
      {\Gread@@xetex@aux#1}%
    }
    \makeatother

    % The hyperref package gives us a pdf with properly built
    % internal navigation ('pdf bookmarks' for the table of contents,
    % internal cross-reference links, web links for URLs, etc.)
    \usepackage{hyperref}
    % The default LaTeX title has an obnoxious amount of whitespace. By default,
    % titling removes some of it. It also provides customization options.
    \usepackage{titling}
    \usepackage{longtable} % longtable support required by pandoc >1.10
    \usepackage{booktabs}  % table support for pandoc > 1.12.2
    \usepackage[inline]{enumitem} % IRkernel/repr support (it uses the enumerate* environment)
    \usepackage[normalem]{ulem} % ulem is needed to support strikethroughs (\sout)
                                % normalem makes italics be italics, not underlines
    \usepackage{mathrsfs}
    

    
    % Colors for the hyperref package
    \definecolor{urlcolor}{rgb}{0,.145,.698}
    \definecolor{linkcolor}{rgb}{.71,0.21,0.01}
    \definecolor{citecolor}{rgb}{.12,.54,.11}

    % ANSI colors
    \definecolor{ansi-black}{HTML}{3E424D}
    \definecolor{ansi-black-intense}{HTML}{282C36}
    \definecolor{ansi-red}{HTML}{E75C58}
    \definecolor{ansi-red-intense}{HTML}{B22B31}
    \definecolor{ansi-green}{HTML}{00A250}
    \definecolor{ansi-green-intense}{HTML}{007427}
    \definecolor{ansi-yellow}{HTML}{DDB62B}
    \definecolor{ansi-yellow-intense}{HTML}{B27D12}
    \definecolor{ansi-blue}{HTML}{208FFB}
    \definecolor{ansi-blue-intense}{HTML}{0065CA}
    \definecolor{ansi-magenta}{HTML}{D160C4}
    \definecolor{ansi-magenta-intense}{HTML}{A03196}
    \definecolor{ansi-cyan}{HTML}{60C6C8}
    \definecolor{ansi-cyan-intense}{HTML}{258F8F}
    \definecolor{ansi-white}{HTML}{C5C1B4}
    \definecolor{ansi-white-intense}{HTML}{A1A6B2}
    \definecolor{ansi-default-inverse-fg}{HTML}{FFFFFF}
    \definecolor{ansi-default-inverse-bg}{HTML}{000000}

    % commands and environments needed by pandoc snippets
    % extracted from the output of `pandoc -s`
    \providecommand{\tightlist}{%
      \setlength{\itemsep}{0pt}\setlength{\parskip}{0pt}}
    \DefineVerbatimEnvironment{Highlighting}{Verbatim}{commandchars=\\\{\}}
    % Add ',fontsize=\small' for more characters per line
    \newenvironment{Shaded}{}{}
    \newcommand{\KeywordTok}[1]{\textcolor[rgb]{0.00,0.44,0.13}{\textbf{{#1}}}}
    \newcommand{\DataTypeTok}[1]{\textcolor[rgb]{0.56,0.13,0.00}{{#1}}}
    \newcommand{\DecValTok}[1]{\textcolor[rgb]{0.25,0.63,0.44}{{#1}}}
    \newcommand{\BaseNTok}[1]{\textcolor[rgb]{0.25,0.63,0.44}{{#1}}}
    \newcommand{\FloatTok}[1]{\textcolor[rgb]{0.25,0.63,0.44}{{#1}}}
    \newcommand{\CharTok}[1]{\textcolor[rgb]{0.25,0.44,0.63}{{#1}}}
    \newcommand{\StringTok}[1]{\textcolor[rgb]{0.25,0.44,0.63}{{#1}}}
    \newcommand{\CommentTok}[1]{\textcolor[rgb]{0.38,0.63,0.69}{\textit{{#1}}}}
    \newcommand{\OtherTok}[1]{\textcolor[rgb]{0.00,0.44,0.13}{{#1}}}
    \newcommand{\AlertTok}[1]{\textcolor[rgb]{1.00,0.00,0.00}{\textbf{{#1}}}}
    \newcommand{\FunctionTok}[1]{\textcolor[rgb]{0.02,0.16,0.49}{{#1}}}
    \newcommand{\RegionMarkerTok}[1]{{#1}}
    \newcommand{\ErrorTok}[1]{\textcolor[rgb]{1.00,0.00,0.00}{\textbf{{#1}}}}
    \newcommand{\NormalTok}[1]{{#1}}
    
    % Additional commands for more recent versions of Pandoc
    \newcommand{\ConstantTok}[1]{\textcolor[rgb]{0.53,0.00,0.00}{{#1}}}
    \newcommand{\SpecialCharTok}[1]{\textcolor[rgb]{0.25,0.44,0.63}{{#1}}}
    \newcommand{\VerbatimStringTok}[1]{\textcolor[rgb]{0.25,0.44,0.63}{{#1}}}
    \newcommand{\SpecialStringTok}[1]{\textcolor[rgb]{0.73,0.40,0.53}{{#1}}}
    \newcommand{\ImportTok}[1]{{#1}}
    \newcommand{\DocumentationTok}[1]{\textcolor[rgb]{0.73,0.13,0.13}{\textit{{#1}}}}
    \newcommand{\AnnotationTok}[1]{\textcolor[rgb]{0.38,0.63,0.69}{\textbf{\textit{{#1}}}}}
    \newcommand{\CommentVarTok}[1]{\textcolor[rgb]{0.38,0.63,0.69}{\textbf{\textit{{#1}}}}}
    \newcommand{\VariableTok}[1]{\textcolor[rgb]{0.10,0.09,0.49}{{#1}}}
    \newcommand{\ControlFlowTok}[1]{\textcolor[rgb]{0.00,0.44,0.13}{\textbf{{#1}}}}
    \newcommand{\OperatorTok}[1]{\textcolor[rgb]{0.40,0.40,0.40}{{#1}}}
    \newcommand{\BuiltInTok}[1]{{#1}}
    \newcommand{\ExtensionTok}[1]{{#1}}
    \newcommand{\PreprocessorTok}[1]{\textcolor[rgb]{0.74,0.48,0.00}{{#1}}}
    \newcommand{\AttributeTok}[1]{\textcolor[rgb]{0.49,0.56,0.16}{{#1}}}
    \newcommand{\InformationTok}[1]{\textcolor[rgb]{0.38,0.63,0.69}{\textbf{\textit{{#1}}}}}
    \newcommand{\WarningTok}[1]{\textcolor[rgb]{0.38,0.63,0.69}{\textbf{\textit{{#1}}}}}
    
    
    % Define a nice break command that doesn't care if a line doesn't already
    % exist.
    \def\br{\hspace*{\fill} \\* }
    % Math Jax compatibility definitions
    \def\gt{>}
    \def\lt{<}
    \let\Oldtex\TeX
    \let\Oldlatex\LaTeX
    \renewcommand{\TeX}{\textrm{\Oldtex}}
    \renewcommand{\LaTeX}{\textrm{\Oldlatex}}
    % Document parameters
    % Document title
    \title{Homework 2}
    
    
    
    
    
% Pygments definitions
\makeatletter
\def\PY@reset{\let\PY@it=\relax \let\PY@bf=\relax%
    \let\PY@ul=\relax \let\PY@tc=\relax%
    \let\PY@bc=\relax \let\PY@ff=\relax}
\def\PY@tok#1{\csname PY@tok@#1\endcsname}
\def\PY@toks#1+{\ifx\relax#1\empty\else%
    \PY@tok{#1}\expandafter\PY@toks\fi}
\def\PY@do#1{\PY@bc{\PY@tc{\PY@ul{%
    \PY@it{\PY@bf{\PY@ff{#1}}}}}}}
\def\PY#1#2{\PY@reset\PY@toks#1+\relax+\PY@do{#2}}

\expandafter\def\csname PY@tok@w\endcsname{\def\PY@tc##1{\textcolor[rgb]{0.73,0.73,0.73}{##1}}}
\expandafter\def\csname PY@tok@c\endcsname{\let\PY@it=\textit\def\PY@tc##1{\textcolor[rgb]{0.25,0.50,0.50}{##1}}}
\expandafter\def\csname PY@tok@cp\endcsname{\def\PY@tc##1{\textcolor[rgb]{0.74,0.48,0.00}{##1}}}
\expandafter\def\csname PY@tok@k\endcsname{\let\PY@bf=\textbf\def\PY@tc##1{\textcolor[rgb]{0.00,0.50,0.00}{##1}}}
\expandafter\def\csname PY@tok@kp\endcsname{\def\PY@tc##1{\textcolor[rgb]{0.00,0.50,0.00}{##1}}}
\expandafter\def\csname PY@tok@kt\endcsname{\def\PY@tc##1{\textcolor[rgb]{0.69,0.00,0.25}{##1}}}
\expandafter\def\csname PY@tok@o\endcsname{\def\PY@tc##1{\textcolor[rgb]{0.40,0.40,0.40}{##1}}}
\expandafter\def\csname PY@tok@ow\endcsname{\let\PY@bf=\textbf\def\PY@tc##1{\textcolor[rgb]{0.67,0.13,1.00}{##1}}}
\expandafter\def\csname PY@tok@nb\endcsname{\def\PY@tc##1{\textcolor[rgb]{0.00,0.50,0.00}{##1}}}
\expandafter\def\csname PY@tok@nf\endcsname{\def\PY@tc##1{\textcolor[rgb]{0.00,0.00,1.00}{##1}}}
\expandafter\def\csname PY@tok@nc\endcsname{\let\PY@bf=\textbf\def\PY@tc##1{\textcolor[rgb]{0.00,0.00,1.00}{##1}}}
\expandafter\def\csname PY@tok@nn\endcsname{\let\PY@bf=\textbf\def\PY@tc##1{\textcolor[rgb]{0.00,0.00,1.00}{##1}}}
\expandafter\def\csname PY@tok@ne\endcsname{\let\PY@bf=\textbf\def\PY@tc##1{\textcolor[rgb]{0.82,0.25,0.23}{##1}}}
\expandafter\def\csname PY@tok@nv\endcsname{\def\PY@tc##1{\textcolor[rgb]{0.10,0.09,0.49}{##1}}}
\expandafter\def\csname PY@tok@no\endcsname{\def\PY@tc##1{\textcolor[rgb]{0.53,0.00,0.00}{##1}}}
\expandafter\def\csname PY@tok@nl\endcsname{\def\PY@tc##1{\textcolor[rgb]{0.63,0.63,0.00}{##1}}}
\expandafter\def\csname PY@tok@ni\endcsname{\let\PY@bf=\textbf\def\PY@tc##1{\textcolor[rgb]{0.60,0.60,0.60}{##1}}}
\expandafter\def\csname PY@tok@na\endcsname{\def\PY@tc##1{\textcolor[rgb]{0.49,0.56,0.16}{##1}}}
\expandafter\def\csname PY@tok@nt\endcsname{\let\PY@bf=\textbf\def\PY@tc##1{\textcolor[rgb]{0.00,0.50,0.00}{##1}}}
\expandafter\def\csname PY@tok@nd\endcsname{\def\PY@tc##1{\textcolor[rgb]{0.67,0.13,1.00}{##1}}}
\expandafter\def\csname PY@tok@s\endcsname{\def\PY@tc##1{\textcolor[rgb]{0.73,0.13,0.13}{##1}}}
\expandafter\def\csname PY@tok@sd\endcsname{\let\PY@it=\textit\def\PY@tc##1{\textcolor[rgb]{0.73,0.13,0.13}{##1}}}
\expandafter\def\csname PY@tok@si\endcsname{\let\PY@bf=\textbf\def\PY@tc##1{\textcolor[rgb]{0.73,0.40,0.53}{##1}}}
\expandafter\def\csname PY@tok@se\endcsname{\let\PY@bf=\textbf\def\PY@tc##1{\textcolor[rgb]{0.73,0.40,0.13}{##1}}}
\expandafter\def\csname PY@tok@sr\endcsname{\def\PY@tc##1{\textcolor[rgb]{0.73,0.40,0.53}{##1}}}
\expandafter\def\csname PY@tok@ss\endcsname{\def\PY@tc##1{\textcolor[rgb]{0.10,0.09,0.49}{##1}}}
\expandafter\def\csname PY@tok@sx\endcsname{\def\PY@tc##1{\textcolor[rgb]{0.00,0.50,0.00}{##1}}}
\expandafter\def\csname PY@tok@m\endcsname{\def\PY@tc##1{\textcolor[rgb]{0.40,0.40,0.40}{##1}}}
\expandafter\def\csname PY@tok@gh\endcsname{\let\PY@bf=\textbf\def\PY@tc##1{\textcolor[rgb]{0.00,0.00,0.50}{##1}}}
\expandafter\def\csname PY@tok@gu\endcsname{\let\PY@bf=\textbf\def\PY@tc##1{\textcolor[rgb]{0.50,0.00,0.50}{##1}}}
\expandafter\def\csname PY@tok@gd\endcsname{\def\PY@tc##1{\textcolor[rgb]{0.63,0.00,0.00}{##1}}}
\expandafter\def\csname PY@tok@gi\endcsname{\def\PY@tc##1{\textcolor[rgb]{0.00,0.63,0.00}{##1}}}
\expandafter\def\csname PY@tok@gr\endcsname{\def\PY@tc##1{\textcolor[rgb]{1.00,0.00,0.00}{##1}}}
\expandafter\def\csname PY@tok@ge\endcsname{\let\PY@it=\textit}
\expandafter\def\csname PY@tok@gs\endcsname{\let\PY@bf=\textbf}
\expandafter\def\csname PY@tok@gp\endcsname{\let\PY@bf=\textbf\def\PY@tc##1{\textcolor[rgb]{0.00,0.00,0.50}{##1}}}
\expandafter\def\csname PY@tok@go\endcsname{\def\PY@tc##1{\textcolor[rgb]{0.53,0.53,0.53}{##1}}}
\expandafter\def\csname PY@tok@gt\endcsname{\def\PY@tc##1{\textcolor[rgb]{0.00,0.27,0.87}{##1}}}
\expandafter\def\csname PY@tok@err\endcsname{\def\PY@bc##1{\setlength{\fboxsep}{0pt}\fcolorbox[rgb]{1.00,0.00,0.00}{1,1,1}{\strut ##1}}}
\expandafter\def\csname PY@tok@kc\endcsname{\let\PY@bf=\textbf\def\PY@tc##1{\textcolor[rgb]{0.00,0.50,0.00}{##1}}}
\expandafter\def\csname PY@tok@kd\endcsname{\let\PY@bf=\textbf\def\PY@tc##1{\textcolor[rgb]{0.00,0.50,0.00}{##1}}}
\expandafter\def\csname PY@tok@kn\endcsname{\let\PY@bf=\textbf\def\PY@tc##1{\textcolor[rgb]{0.00,0.50,0.00}{##1}}}
\expandafter\def\csname PY@tok@kr\endcsname{\let\PY@bf=\textbf\def\PY@tc##1{\textcolor[rgb]{0.00,0.50,0.00}{##1}}}
\expandafter\def\csname PY@tok@bp\endcsname{\def\PY@tc##1{\textcolor[rgb]{0.00,0.50,0.00}{##1}}}
\expandafter\def\csname PY@tok@fm\endcsname{\def\PY@tc##1{\textcolor[rgb]{0.00,0.00,1.00}{##1}}}
\expandafter\def\csname PY@tok@vc\endcsname{\def\PY@tc##1{\textcolor[rgb]{0.10,0.09,0.49}{##1}}}
\expandafter\def\csname PY@tok@vg\endcsname{\def\PY@tc##1{\textcolor[rgb]{0.10,0.09,0.49}{##1}}}
\expandafter\def\csname PY@tok@vi\endcsname{\def\PY@tc##1{\textcolor[rgb]{0.10,0.09,0.49}{##1}}}
\expandafter\def\csname PY@tok@vm\endcsname{\def\PY@tc##1{\textcolor[rgb]{0.10,0.09,0.49}{##1}}}
\expandafter\def\csname PY@tok@sa\endcsname{\def\PY@tc##1{\textcolor[rgb]{0.73,0.13,0.13}{##1}}}
\expandafter\def\csname PY@tok@sb\endcsname{\def\PY@tc##1{\textcolor[rgb]{0.73,0.13,0.13}{##1}}}
\expandafter\def\csname PY@tok@sc\endcsname{\def\PY@tc##1{\textcolor[rgb]{0.73,0.13,0.13}{##1}}}
\expandafter\def\csname PY@tok@dl\endcsname{\def\PY@tc##1{\textcolor[rgb]{0.73,0.13,0.13}{##1}}}
\expandafter\def\csname PY@tok@s2\endcsname{\def\PY@tc##1{\textcolor[rgb]{0.73,0.13,0.13}{##1}}}
\expandafter\def\csname PY@tok@sh\endcsname{\def\PY@tc##1{\textcolor[rgb]{0.73,0.13,0.13}{##1}}}
\expandafter\def\csname PY@tok@s1\endcsname{\def\PY@tc##1{\textcolor[rgb]{0.73,0.13,0.13}{##1}}}
\expandafter\def\csname PY@tok@mb\endcsname{\def\PY@tc##1{\textcolor[rgb]{0.40,0.40,0.40}{##1}}}
\expandafter\def\csname PY@tok@mf\endcsname{\def\PY@tc##1{\textcolor[rgb]{0.40,0.40,0.40}{##1}}}
\expandafter\def\csname PY@tok@mh\endcsname{\def\PY@tc##1{\textcolor[rgb]{0.40,0.40,0.40}{##1}}}
\expandafter\def\csname PY@tok@mi\endcsname{\def\PY@tc##1{\textcolor[rgb]{0.40,0.40,0.40}{##1}}}
\expandafter\def\csname PY@tok@il\endcsname{\def\PY@tc##1{\textcolor[rgb]{0.40,0.40,0.40}{##1}}}
\expandafter\def\csname PY@tok@mo\endcsname{\def\PY@tc##1{\textcolor[rgb]{0.40,0.40,0.40}{##1}}}
\expandafter\def\csname PY@tok@ch\endcsname{\let\PY@it=\textit\def\PY@tc##1{\textcolor[rgb]{0.25,0.50,0.50}{##1}}}
\expandafter\def\csname PY@tok@cm\endcsname{\let\PY@it=\textit\def\PY@tc##1{\textcolor[rgb]{0.25,0.50,0.50}{##1}}}
\expandafter\def\csname PY@tok@cpf\endcsname{\let\PY@it=\textit\def\PY@tc##1{\textcolor[rgb]{0.25,0.50,0.50}{##1}}}
\expandafter\def\csname PY@tok@c1\endcsname{\let\PY@it=\textit\def\PY@tc##1{\textcolor[rgb]{0.25,0.50,0.50}{##1}}}
\expandafter\def\csname PY@tok@cs\endcsname{\let\PY@it=\textit\def\PY@tc##1{\textcolor[rgb]{0.25,0.50,0.50}{##1}}}

\def\PYZbs{\char`\\}
\def\PYZus{\char`\_}
\def\PYZob{\char`\{}
\def\PYZcb{\char`\}}
\def\PYZca{\char`\^}
\def\PYZam{\char`\&}
\def\PYZlt{\char`\<}
\def\PYZgt{\char`\>}
\def\PYZsh{\char`\#}
\def\PYZpc{\char`\%}
\def\PYZdl{\char`\$}
\def\PYZhy{\char`\-}
\def\PYZsq{\char`\'}
\def\PYZdq{\char`\"}
\def\PYZti{\char`\~}
% for compatibility with earlier versions
\def\PYZat{@}
\def\PYZlb{[}
\def\PYZrb{]}
\makeatother


    % For linebreaks inside Verbatim environment from package fancyvrb. 
    \makeatletter
        \newbox\Wrappedcontinuationbox 
        \newbox\Wrappedvisiblespacebox 
        \newcommand*\Wrappedvisiblespace {\textcolor{red}{\textvisiblespace}} 
        \newcommand*\Wrappedcontinuationsymbol {\textcolor{red}{\llap{\tiny$\m@th\hookrightarrow$}}} 
        \newcommand*\Wrappedcontinuationindent {3ex } 
        \newcommand*\Wrappedafterbreak {\kern\Wrappedcontinuationindent\copy\Wrappedcontinuationbox} 
        % Take advantage of the already applied Pygments mark-up to insert 
        % potential linebreaks for TeX processing. 
        %        {, <, #, %, $, ' and ": go to next line. 
        %        _, }, ^, &, >, - and ~: stay at end of broken line. 
        % Use of \textquotesingle for straight quote. 
        \newcommand*\Wrappedbreaksatspecials {% 
            \def\PYGZus{\discretionary{\char`\_}{\Wrappedafterbreak}{\char`\_}}% 
            \def\PYGZob{\discretionary{}{\Wrappedafterbreak\char`\{}{\char`\{}}% 
            \def\PYGZcb{\discretionary{\char`\}}{\Wrappedafterbreak}{\char`\}}}% 
            \def\PYGZca{\discretionary{\char`\^}{\Wrappedafterbreak}{\char`\^}}% 
            \def\PYGZam{\discretionary{\char`\&}{\Wrappedafterbreak}{\char`\&}}% 
            \def\PYGZlt{\discretionary{}{\Wrappedafterbreak\char`\<}{\char`\<}}% 
            \def\PYGZgt{\discretionary{\char`\>}{\Wrappedafterbreak}{\char`\>}}% 
            \def\PYGZsh{\discretionary{}{\Wrappedafterbreak\char`\#}{\char`\#}}% 
            \def\PYGZpc{\discretionary{}{\Wrappedafterbreak\char`\%}{\char`\%}}% 
            \def\PYGZdl{\discretionary{}{\Wrappedafterbreak\char`\$}{\char`\$}}% 
            \def\PYGZhy{\discretionary{\char`\-}{\Wrappedafterbreak}{\char`\-}}% 
            \def\PYGZsq{\discretionary{}{\Wrappedafterbreak\textquotesingle}{\textquotesingle}}% 
            \def\PYGZdq{\discretionary{}{\Wrappedafterbreak\char`\"}{\char`\"}}% 
            \def\PYGZti{\discretionary{\char`\~}{\Wrappedafterbreak}{\char`\~}}% 
        } 
        % Some characters . , ; ? ! / are not pygmentized. 
        % This macro makes them "active" and they will insert potential linebreaks 
        \newcommand*\Wrappedbreaksatpunct {% 
            \lccode`\~`\.\lowercase{\def~}{\discretionary{\hbox{\char`\.}}{\Wrappedafterbreak}{\hbox{\char`\.}}}% 
            \lccode`\~`\,\lowercase{\def~}{\discretionary{\hbox{\char`\,}}{\Wrappedafterbreak}{\hbox{\char`\,}}}% 
            \lccode`\~`\;\lowercase{\def~}{\discretionary{\hbox{\char`\;}}{\Wrappedafterbreak}{\hbox{\char`\;}}}% 
            \lccode`\~`\:\lowercase{\def~}{\discretionary{\hbox{\char`\:}}{\Wrappedafterbreak}{\hbox{\char`\:}}}% 
            \lccode`\~`\?\lowercase{\def~}{\discretionary{\hbox{\char`\?}}{\Wrappedafterbreak}{\hbox{\char`\?}}}% 
            \lccode`\~`\!\lowercase{\def~}{\discretionary{\hbox{\char`\!}}{\Wrappedafterbreak}{\hbox{\char`\!}}}% 
            \lccode`\~`\/\lowercase{\def~}{\discretionary{\hbox{\char`\/}}{\Wrappedafterbreak}{\hbox{\char`\/}}}% 
            \catcode`\.\active
            \catcode`\,\active 
            \catcode`\;\active
            \catcode`\:\active
            \catcode`\?\active
            \catcode`\!\active
            \catcode`\/\active 
            \lccode`\~`\~ 	
        }
    \makeatother

    \let\OriginalVerbatim=\Verbatim
    \makeatletter
    \renewcommand{\Verbatim}[1][1]{%
        %\parskip\z@skip
        \sbox\Wrappedcontinuationbox {\Wrappedcontinuationsymbol}%
        \sbox\Wrappedvisiblespacebox {\FV@SetupFont\Wrappedvisiblespace}%
        \def\FancyVerbFormatLine ##1{\hsize\linewidth
            \vtop{\raggedright\hyphenpenalty\z@\exhyphenpenalty\z@
                \doublehyphendemerits\z@\finalhyphendemerits\z@
                \strut ##1\strut}%
        }%
        % If the linebreak is at a space, the latter will be displayed as visible
        % space at end of first line, and a continuation symbol starts next line.
        % Stretch/shrink are however usually zero for typewriter font.
        \def\FV@Space {%
            \nobreak\hskip\z@ plus\fontdimen3\font minus\fontdimen4\font
            \discretionary{\copy\Wrappedvisiblespacebox}{\Wrappedafterbreak}
            {\kern\fontdimen2\font}%
        }%
        
        % Allow breaks at special characters using \PYG... macros.
        \Wrappedbreaksatspecials
        % Breaks at punctuation characters . , ; ? ! and / need catcode=\active 	
        \OriginalVerbatim[#1,codes*=\Wrappedbreaksatpunct]%
    }
    \makeatother

    % Exact colors from NB
    \definecolor{incolor}{HTML}{303F9F}
    \definecolor{outcolor}{HTML}{D84315}
    \definecolor{cellborder}{HTML}{CFCFCF}
    \definecolor{cellbackground}{HTML}{F7F7F7}
    
    % prompt
    \makeatletter
    \newcommand{\boxspacing}{\kern\kvtcb@left@rule\kern\kvtcb@boxsep}
    \makeatother
    \newcommand{\prompt}[4]{
        \ttfamily\llap{{\color{#2}[#3]:\hspace{3pt}#4}}\vspace{-\baselineskip}
    }
    

    
    % Prevent overflowing lines due to hard-to-break entities
    \sloppy 
    % Setup hyperref package
    \hypersetup{
      breaklinks=true,  % so long urls are correctly broken across lines
      colorlinks=true,
      urlcolor=urlcolor,
      linkcolor=linkcolor,
      citecolor=citecolor,
      }
    % Slightly bigger margins than the latex defaults
    
    \geometry{verbose,tmargin=1in,bmargin=1in,lmargin=1in,rmargin=1in}
    
    

\begin{document}
    
    \maketitle
    
    

    
    Motor Design Problem / Heather Miller / Started: 4/22/20

    \begin{tcolorbox}[breakable, size=fbox, boxrule=1pt, pad at break*=1mm,colback=cellbackground, colframe=cellborder]
\prompt{In}{incolor}{3}{\boxspacing}
\begin{Verbatim}[commandchars=\\\{\}]
\PY{k+kn}{import} \PY{n+nn}{numpy} \PY{k}{as} \PY{n+nn}{np}
\PY{k+kn}{from} \PY{n+nn}{scipy}\PY{n+nn}{.}\PY{n+nn}{stats} \PY{k+kn}{import} \PY{n}{norm}
\end{Verbatim}
\end{tcolorbox}

    Motor Design problem : weight of motor must be \textless{}22kgs

    \begin{tcolorbox}[breakable, size=fbox, boxrule=1pt, pad at break*=1mm,colback=cellbackground, colframe=cellborder]
\prompt{In}{incolor}{23}{\boxspacing}
\begin{Verbatim}[commandchars=\\\{\}]
\PY{c+c1}{\PYZsh{} answers for each part will be stored here}
\PY{n}{success\PYZus{}probabilities} \PY{o}{=} \PY{p}{[}\PY{p}{]}

\PY{c+c1}{\PYZsh{} number of samples to run}
\PY{n}{samples} \PY{o}{=} \PY{l+m+mi}{100000}

\PY{c+c1}{\PYZsh{}limit in kgs}
\PY{n}{limit} \PY{o}{=} \PY{l+m+mi}{22} 
\end{Verbatim}
\end{tcolorbox}

    Given Constraints

    \begin{tcolorbox}[breakable, size=fbox, boxrule=1pt, pad at break*=1mm,colback=cellbackground, colframe=cellborder]
\prompt{In}{incolor}{24}{\boxspacing}
\begin{Verbatim}[commandchars=\\\{\}]
\PY{c+c1}{\PYZsh{} Motor Design Variables}
\PY{n}{Lwa} \PY{o}{=} \PY{l+m+mf}{14.12}  \PY{c+c1}{\PYZsh{} armature wire length (m)}
\PY{n}{Lwf} \PY{o}{=} \PY{l+m+mf}{309.45}  \PY{c+c1}{\PYZsh{} field wire length (m)}
\PY{n}{ds} \PY{o}{=} \PY{l+m+mf}{0.00612}  \PY{c+c1}{\PYZsh{} slot depth (m)}

\PY{c+c1}{\PYZsh{} Coupling Variables b, shared with control problem}
\PY{n}{n} \PY{o}{=} \PY{l+m+mi}{122}  \PY{c+c1}{\PYZsh{} rotational speed (rev/s)}
\PY{n}{v} \PY{o}{=} \PY{l+m+mi}{40}  \PY{c+c1}{\PYZsh{} design voltage (V)}
\PY{n}{pmin} \PY{o}{=} \PY{l+m+mf}{3.94}  \PY{c+c1}{\PYZsh{} minimum required power (kW)}
\PY{n}{ymin} \PY{o}{=} \PY{l+m+mf}{5.12e\PYZhy{}3}  \PY{c+c1}{\PYZsh{} minimum required torque (kNm)}

\PY{c+c1}{\PYZsh{} Parameter Vector a (constants)}
\PY{n}{fi} \PY{o}{=} \PY{l+m+mf}{0.7}  \PY{c+c1}{\PYZsh{} pole arc to pole pitch ratio}
\PY{n}{p} \PY{o}{=} \PY{l+m+mi}{2}  \PY{c+c1}{\PYZsh{} number of poles}
\PY{n}{s} \PY{o}{=} \PY{l+m+mi}{27}  \PY{c+c1}{\PYZsh{} number of slots (teeth on rotor)}
\PY{n}{rho} \PY{o}{=} \PY{l+m+mf}{1.8e\PYZhy{}8}  \PY{c+c1}{\PYZsh{} resistivity (ohm\PYZhy{}m) of copper at 25C}

\PY{c+c1}{\PYZsh{} Derived parameters and constants}
\PY{n}{mu0} \PY{o}{=} \PY{l+m+mi}{4} \PY{o}{*} \PY{n}{np}\PY{o}{.}\PY{n}{pi} \PY{o}{*} \PY{l+m+mf}{1e\PYZhy{}7}  \PY{c+c1}{\PYZsh{} magnetic constant}
\PY{n}{ap} \PY{o}{=} \PY{n}{p}  \PY{c+c1}{\PYZsh{} parallel circuit paths (equals poles)}
\PY{n}{eff} \PY{o}{=} \PY{l+m+mf}{0.85}  \PY{c+c1}{\PYZsh{} efficiency}
\PY{n}{bfc} \PY{o}{=} \PY{l+m+mf}{40e\PYZhy{}3}  \PY{c+c1}{\PYZsh{} pole depth (m)}
\PY{n}{fcf} \PY{o}{=} \PY{l+m+mf}{0.55}  \PY{c+c1}{\PYZsh{} field coil spacing factor}
\PY{n}{Awa} \PY{o}{=} \PY{l+m+mf}{2.0074e\PYZhy{}006}  \PY{c+c1}{\PYZsh{} cross sectional area of armature winding (m\PYZca{}2)}
\PY{n}{Awf} \PY{o}{=} \PY{l+m+mf}{0.2749e\PYZhy{}6}  \PY{c+c1}{\PYZsh{} cross sectional area of field coil winding (m\PYZca{}2)}
\end{Verbatim}
\end{tcolorbox}

    Functions used in this code

    \begin{tcolorbox}[breakable, size=fbox, boxrule=1pt, pad at break*=1mm,colback=cellbackground, colframe=cellborder]
\prompt{In}{incolor}{25}{\boxspacing}
\begin{Verbatim}[commandchars=\\\{\}]
\PY{k}{def} \PY{n+nf}{calculate\PYZus{}weight}\PY{p}{(}\PY{n}{diameter}\PY{p}{,} \PY{n}{length}\PY{p}{,} \PY{n}{rho\PYZus{}cu}\PY{p}{,} \PY{n}{rho\PYZus{}fe}\PY{p}{)}\PY{p}{:}
    \PY{c+c1}{\PYZsh{} calculate the weight of the motor}
    \PY{n}{weight} \PY{o}{=} \PY{n}{rho\PYZus{}cu} \PY{o}{*} \PY{p}{(}\PY{n}{Awa}\PY{o}{*}\PY{n}{Lwa} \PY{o}{+} \PY{n}{Awf}\PY{o}{*}\PY{n}{Lwf}\PY{p}{)} \PY{o}{+} \PY{n}{rho\PYZus{}fe} \PY{o}{*} \PY{n}{length} \PY{o}{*} \PY{n}{np}\PY{o}{.}\PY{n}{pi} \PY{o}{*} \PY{n+nb}{pow}\PY{p}{(}\PY{n}{diameter} \PY{o}{+} \PY{n}{ds}\PY{p}{,} \PY{l+m+mi}{2}\PY{p}{)}
    \PY{k}{return} \PY{n}{weight}

\PY{k}{def} \PY{n+nf}{prob\PYZus{}success\PYZus{}mc}\PY{p}{(}\PY{n}{weights}\PY{p}{,} \PY{n}{limit}\PY{p}{)}\PY{p}{:}
    \PY{c+c1}{\PYZsh{} determine the probability of success that the weight of engine will be less than a limit}
    \PY{c+c1}{\PYZsh{} add 1 to limit\PYZus{}sum every time weight is under limit weight}
    \PY{n}{x} \PY{o}{=} \PY{n+nb}{len}\PY{p}{(}\PY{n}{weights}\PY{p}{)}
    \PY{n}{limit\PYZus{}sum} \PY{o}{=} \PY{l+m+mi}{0}
    \PY{k}{for} \PY{n}{i} \PY{o+ow}{in} \PY{n}{weights}\PY{p}{:}
        \PY{k}{if} \PY{n}{i} \PY{o}{\PYZlt{}} \PY{n}{limit}\PY{p}{:}
            \PY{n}{limit\PYZus{}sum} \PY{o}{+}\PY{o}{=} \PY{l+m+mi}{1}
    \PY{k}{return} \PY{n}{limit\PYZus{}sum}\PY{o}{/}\PY{n}{x}


\PY{k}{def} \PY{n+nf}{first\PYZus{}derivative}\PY{p}{(}\PY{n}{variable\PYZus{}dict}\PY{p}{,} \PY{n}{variable\PYZus{}of\PYZus{}interest}\PY{p}{)}\PY{p}{:}
    \PY{c+c1}{\PYZsh{} calculate the first derivatives of each of the variables}
    \PY{n}{h} \PY{o}{=} \PY{l+m+mf}{0.1} \PY{o}{*} \PY{n}{variable\PYZus{}dict}\PY{p}{[}\PY{n}{variable\PYZus{}of\PYZus{}interest}\PY{p}{]}\PY{p}{[}\PY{l+m+mi}{1}\PY{p}{]}
    \PY{c+c1}{\PYZsh{} inputs [diameter, length, rho\PYZus{}cu, rho\PYZus{}fe]}
    \PY{n}{inputs} \PY{o}{=} \PY{p}{[}\PY{p}{]}
    \PY{c+c1}{\PYZsh{} this loop will put two values into input for each value, if the variable selected matches the key it will}
    \PY{c+c1}{\PYZsh{} modify those values with h otherwise both inputs will be the same}
    \PY{k}{for} \PY{n}{key} \PY{o+ow}{in} \PY{n}{variable\PYZus{}dict}\PY{p}{:}
        \PY{k}{if} \PY{n}{key} \PY{o}{==} \PY{n}{variable\PYZus{}of\PYZus{}interest}\PY{p}{:}
            \PY{n}{inputs}\PY{o}{.}\PY{n}{append}\PY{p}{(}\PY{p}{[}\PY{n}{variable\PYZus{}dict}\PY{p}{[}\PY{n}{key}\PY{p}{]}\PY{p}{[}\PY{l+m+mi}{0}\PY{p}{]} \PY{o}{+} \PY{n}{h}\PY{p}{,} \PY{n}{variable\PYZus{}dict}\PY{p}{[}\PY{n}{key}\PY{p}{]}\PY{p}{[}\PY{l+m+mi}{0}\PY{p}{]} \PY{o}{\PYZhy{}} \PY{n}{h}\PY{p}{]}\PY{p}{)}
        \PY{k}{else}\PY{p}{:}
            \PY{n}{inputs}\PY{o}{.}\PY{n}{append}\PY{p}{(}\PY{p}{[}\PY{n}{variable\PYZus{}dict}\PY{p}{[}\PY{n}{key}\PY{p}{]}\PY{p}{[}\PY{l+m+mi}{0}\PY{p}{]}\PY{p}{,} \PY{n}{variable\PYZus{}dict}\PY{p}{[}\PY{n}{key}\PY{p}{]}\PY{p}{[}\PY{l+m+mi}{0}\PY{p}{]}\PY{p}{]}\PY{p}{)}
    \PY{c+c1}{\PYZsh{} calculate the first derivative with the values in the input}
    \PY{n}{first\PYZus{}d} \PY{o}{=} \PY{p}{(}\PY{n}{calculate\PYZus{}weight}\PY{p}{(}\PY{n}{inputs}\PY{p}{[}\PY{l+m+mi}{0}\PY{p}{]}\PY{p}{[}\PY{l+m+mi}{0}\PY{p}{]}\PY{p}{,} \PY{n}{inputs}\PY{p}{[}\PY{l+m+mi}{1}\PY{p}{]}\PY{p}{[}\PY{l+m+mi}{0}\PY{p}{]}\PY{p}{,} \PY{n}{inputs}\PY{p}{[}\PY{l+m+mi}{2}\PY{p}{]}\PY{p}{[}\PY{l+m+mi}{0}\PY{p}{]}\PY{p}{,} \PY{n}{inputs}\PY{p}{[}\PY{l+m+mi}{3}\PY{p}{]}\PY{p}{[}\PY{l+m+mi}{0}\PY{p}{]}\PY{p}{)} \PY{o}{\PYZhy{}}
               \PY{n}{calculate\PYZus{}weight}\PY{p}{(}\PY{n}{inputs}\PY{p}{[}\PY{l+m+mi}{0}\PY{p}{]}\PY{p}{[}\PY{l+m+mi}{1}\PY{p}{]}\PY{p}{,} \PY{n}{inputs}\PY{p}{[}\PY{l+m+mi}{1}\PY{p}{]}\PY{p}{[}\PY{l+m+mi}{1}\PY{p}{]}\PY{p}{,} \PY{n}{inputs}\PY{p}{[}\PY{l+m+mi}{2}\PY{p}{]}\PY{p}{[}\PY{l+m+mi}{1}\PY{p}{]}\PY{p}{,} \PY{n}{inputs}\PY{p}{[}\PY{l+m+mi}{3}\PY{p}{]}\PY{p}{[}\PY{l+m+mi}{1}\PY{p}{]}\PY{p}{)}\PY{p}{)}\PY{o}{/}\PY{l+m+mi}{2}\PY{o}{*}\PY{n}{h}
    \PY{k}{return} \PY{n}{first\PYZus{}d}

\PY{k}{def} \PY{n+nf}{calculate\PYZus{}sigma}\PY{p}{(}\PY{n}{variable\PYZus{}sigmas}\PY{p}{,} \PY{n}{variable\PYZus{}derivatives}\PY{p}{,} \PY{n}{correlation\PYZus{}matrix}\PY{p}{)}\PY{p}{:}
    \PY{c+c1}{\PYZsh{} calculate the sigma of the function}
    \PY{n}{sigma\PYZus{}squared} \PY{o}{=} \PY{l+m+mi}{0}
    \PY{k}{for} \PY{n}{i} \PY{o+ow}{in} \PY{n+nb}{range}\PY{p}{(}\PY{n+nb}{len}\PY{p}{(}\PY{n}{variable\PYZus{}derivatives}\PY{p}{)}\PY{p}{)}\PY{p}{:}
        \PY{k}{for} \PY{n}{j} \PY{o+ow}{in} \PY{n+nb}{range}\PY{p}{(}\PY{n+nb}{len}\PY{p}{(}\PY{n}{variable\PYZus{}derivatives}\PY{p}{)}\PY{p}{)}\PY{p}{:}
            \PY{n}{sigma\PYZus{}squared} \PY{o}{+}\PY{o}{=} \PY{n}{variable\PYZus{}derivatives}\PY{p}{[}\PY{n}{i}\PY{p}{]} \PY{o}{*} \PY{n}{variable\PYZus{}derivatives}\PY{p}{[}\PY{n}{j}\PY{p}{]} \PY{o}{*} \PY{n}{correlation\PYZus{}matrix}\PY{p}{[}\PY{n}{i}\PY{p}{]}\PY{p}{[}\PY{n}{j}\PY{p}{]} \PYZbs{}
                             \PY{o}{*} \PY{n}{variable\PYZus{}sigmas}\PY{p}{[}\PY{n}{i}\PY{p}{]} \PY{o}{*} \PY{n}{variable\PYZus{}sigmas}\PY{p}{[}\PY{n}{j}\PY{p}{]}
    \PY{k}{return} \PY{n}{np}\PY{o}{.}\PY{n}{sqrt}\PY{p}{(}\PY{n}{sigma\PYZus{}squared}\PY{p}{)}
\end{Verbatim}
\end{tcolorbox}

    \hypertarget{using-monte-carlo-and-normal-distribution}{%
\section{Using Monte Carlo and Normal
Distribution}\label{using-monte-carlo-and-normal-distribution}}

\begin{itemize}
\tightlist
\item
  D\textasciitilde{}N(7.5, 0.5) \%m rotor diameter
\item
  L\textasciitilde{}N(9.5, 0.5) \%m rotor axial length
\item
  dcu\textasciitilde{}N(8.94e3, 100) \%copper density density at 25C
  (kg/m\^{}3)
\item
  dfe\textasciitilde{}N(7.98e3, 100) \%iron density density at 25C
  (kg/m\^{}3)
\end{itemize}

    \begin{tcolorbox}[breakable, size=fbox, boxrule=1pt, pad at break*=1mm,colback=cellbackground, colframe=cellborder]
\prompt{In}{incolor}{26}{\boxspacing}
\begin{Verbatim}[commandchars=\\\{\}]
\PY{n}{d\PYZus{}n} \PY{o}{=} \PY{n}{np}\PY{o}{.}\PY{n}{random}\PY{o}{.}\PY{n}{normal}\PY{p}{(}\PY{l+m+mf}{0.075}\PY{p}{,} \PY{l+m+mf}{0.005}\PY{p}{,} \PY{n}{samples}\PY{p}{)}  \PY{c+c1}{\PYZsh{} rotor diameter (cm)}
\PY{n}{l\PYZus{}n} \PY{o}{=} \PY{n}{np}\PY{o}{.}\PY{n}{random}\PY{o}{.}\PY{n}{normal}\PY{p}{(}\PY{l+m+mf}{0.095}\PY{p}{,} \PY{l+m+mf}{0.005}\PY{p}{,} \PY{n}{samples}\PY{p}{)}  \PY{c+c1}{\PYZsh{} rotor axial length (cm)}
\PY{n}{rho\PYZus{}cu\PYZus{}n} \PY{o}{=} \PY{n}{np}\PY{o}{.}\PY{n}{random}\PY{o}{.}\PY{n}{normal}\PY{p}{(}\PY{l+m+mf}{8.94e3}\PY{p}{,} \PY{l+m+mi}{100}\PY{p}{,} \PY{n}{samples}\PY{p}{)}  \PY{c+c1}{\PYZsh{} copper density density at 25C (kg/m\PYZca{}3)}
\PY{n}{rho\PYZus{}fe\PYZus{}n} \PY{o}{=} \PY{n}{np}\PY{o}{.}\PY{n}{random}\PY{o}{.}\PY{n}{normal}\PY{p}{(}\PY{l+m+mf}{7.98e3}\PY{p}{,} \PY{l+m+mi}{100}\PY{p}{,} \PY{n}{samples}\PY{p}{)}  \PY{c+c1}{\PYZsh{} iron density density at 25C (kg/m\PYZca{}3)}
\PY{n}{weight\PYZus{}p1} \PY{o}{=} \PY{p}{[}\PY{n}{calculate\PYZus{}weight}\PY{p}{(}\PY{n}{d\PYZus{}n}\PY{p}{[}\PY{n}{i}\PY{p}{]}\PY{p}{,} \PY{n}{l\PYZus{}n}\PY{p}{[}\PY{n}{i}\PY{p}{]}\PY{p}{,} \PY{n}{rho\PYZus{}cu\PYZus{}n}\PY{p}{[}\PY{n}{i}\PY{p}{]}\PY{p}{,} \PY{n}{rho\PYZus{}fe\PYZus{}n}\PY{p}{[}\PY{n}{i}\PY{p}{]}\PY{p}{)} \PY{k}{for} \PY{n}{i} \PY{o+ow}{in} \PY{n+nb}{range}\PY{p}{(}\PY{n}{samples}\PY{p}{)}\PY{p}{]}
\PY{n}{success\PYZus{}probabilities}\PY{o}{.}\PY{n}{append}\PY{p}{(}\PY{n}{prob\PYZus{}success\PYZus{}mc}\PY{p}{(}\PY{n}{weight\PYZus{}p1}\PY{p}{,} \PY{n}{limit}\PY{p}{)}\PY{p}{)}
\end{Verbatim}
\end{tcolorbox}

    \hypertarget{using-monte-carlo-and-uniform-distribution}{%
\section{Using Monte Carlo and Uniform
Distribution}\label{using-monte-carlo-and-uniform-distribution}}

\begin{itemize}
\tightlist
\item
  D\textasciitilde{}Uniform(6.5, 8.5) \%m rotor diameter
\item
  L\textasciitilde{}Uniform (8.5, 10.5) \%m rotor axial length
\item
  dcu\textasciitilde{}Uniform (8840, 9040) \%copper density density at
  25C (kg/m\^{}3)
\item
  dfe\textasciitilde{}Uniform (7880, 8080) \%iron density density at 25C
  (kg/m\^{}3)
\end{itemize}

    \begin{tcolorbox}[breakable, size=fbox, boxrule=1pt, pad at break*=1mm,colback=cellbackground, colframe=cellborder]
\prompt{In}{incolor}{27}{\boxspacing}
\begin{Verbatim}[commandchars=\\\{\}]
\PY{n}{d\PYZus{}u} \PY{o}{=} \PY{n}{np}\PY{o}{.}\PY{n}{random}\PY{o}{.}\PY{n}{uniform}\PY{p}{(}\PY{l+m+mf}{0.065}\PY{p}{,} \PY{l+m+mf}{0.085}\PY{p}{,} \PY{n}{samples}\PY{p}{)}  \PY{c+c1}{\PYZsh{} rotor diameter (cm)}
\PY{n}{l\PYZus{}u} \PY{o}{=} \PY{n}{np}\PY{o}{.}\PY{n}{random}\PY{o}{.}\PY{n}{uniform}\PY{p}{(}\PY{l+m+mf}{0.085}\PY{p}{,} \PY{l+m+mf}{0.105}\PY{p}{,} \PY{n}{samples}\PY{p}{)}  \PY{c+c1}{\PYZsh{} rotor axial length (cm)}
\PY{n}{rho\PYZus{}cu\PYZus{}u} \PY{o}{=} \PY{n}{np}\PY{o}{.}\PY{n}{random}\PY{o}{.}\PY{n}{uniform}\PY{p}{(}\PY{l+m+mi}{8840}\PY{p}{,} \PY{l+m+mi}{9040}\PY{p}{,} \PY{n}{samples}\PY{p}{)}  \PY{c+c1}{\PYZsh{} copper density density at 25C (kg/m\PYZca{}3)}
\PY{n}{rho\PYZus{}fe\PYZus{}u} \PY{o}{=} \PY{n}{np}\PY{o}{.}\PY{n}{random}\PY{o}{.}\PY{n}{uniform}\PY{p}{(}\PY{l+m+mi}{7880}\PY{p}{,} \PY{l+m+mi}{8080}\PY{p}{,} \PY{n}{samples}\PY{p}{)}  \PY{c+c1}{\PYZsh{} iron density density at 25C (kg/m\PYZca{}3)}
\PY{n}{weight\PYZus{}p2} \PY{o}{=} \PY{p}{[}\PY{n}{calculate\PYZus{}weight}\PY{p}{(}\PY{n}{d\PYZus{}u}\PY{p}{[}\PY{n}{i}\PY{p}{]}\PY{p}{,} \PY{n}{l\PYZus{}u}\PY{p}{[}\PY{n}{i}\PY{p}{]}\PY{p}{,} \PY{n}{rho\PYZus{}cu\PYZus{}u}\PY{p}{[}\PY{n}{i}\PY{p}{]}\PY{p}{,} \PY{n}{rho\PYZus{}fe\PYZus{}u}\PY{p}{[}\PY{n}{i}\PY{p}{]}\PY{p}{)} \PY{k}{for} \PY{n}{i} \PY{o+ow}{in} \PY{n+nb}{range}\PY{p}{(}\PY{n}{samples}\PY{p}{)}\PY{p}{]}
\PY{n}{success\PYZus{}probabilities}\PY{o}{.}\PY{n}{append}\PY{p}{(}\PY{n}{prob\PYZus{}success\PYZus{}mc}\PY{p}{(}\PY{n}{weight\PYZus{}p2}\PY{p}{,} \PY{n}{limit}\PY{p}{)}\PY{p}{)}
\end{Verbatim}
\end{tcolorbox}

    \hypertarget{using-mvfosm-method}{%
\section{Using MVFOSM method}\label{using-mvfosm-method}}

    \begin{tcolorbox}[breakable, size=fbox, boxrule=1pt, pad at break*=1mm,colback=cellbackground, colframe=cellborder]
\prompt{In}{incolor}{28}{\boxspacing}
\begin{Verbatim}[commandchars=\\\{\}]
\PY{c+c1}{\PYZsh{} no correlation}
\PY{n}{correlation\PYZus{}matrix\PYZus{}3} \PY{o}{=} \PY{p}{[}\PY{p}{[}\PY{l+m+mi}{1}\PY{p}{,} \PY{l+m+mi}{0}\PY{p}{,} \PY{l+m+mi}{0} \PY{p}{,} \PY{l+m+mi}{0}\PY{p}{]}\PY{p}{,} \PY{p}{[}\PY{l+m+mi}{0}\PY{p}{,} \PY{l+m+mi}{1}\PY{p}{,} \PY{l+m+mi}{0}\PY{p}{,} \PY{l+m+mi}{0}\PY{p}{]}\PY{p}{,} \PY{p}{[}\PY{l+m+mi}{0}\PY{p}{,} \PY{l+m+mi}{0}\PY{p}{,} \PY{l+m+mi}{1}\PY{p}{,} \PY{l+m+mi}{0}\PY{p}{]}\PY{p}{,} \PY{p}{[}\PY{l+m+mi}{0}\PY{p}{,} \PY{l+m+mi}{0}\PY{p}{,} \PY{l+m+mi}{0}\PY{p}{,} \PY{l+m+mi}{1}\PY{p}{]}\PY{p}{]}

\PY{n}{variables} \PY{o}{=} \PY{p}{\PYZob{}}\PY{l+s+s2}{\PYZdq{}}\PY{l+s+s2}{diameter}\PY{l+s+s2}{\PYZdq{}}\PY{p}{:} \PY{p}{(}\PY{l+m+mf}{0.075}\PY{p}{,} \PY{l+m+mf}{0.005}\PY{p}{)}\PY{p}{,}
             \PY{l+s+s2}{\PYZdq{}}\PY{l+s+s2}{length}\PY{l+s+s2}{\PYZdq{}}\PY{p}{:} \PY{p}{(}\PY{l+m+mf}{0.095}\PY{p}{,} \PY{l+m+mf}{0.005}\PY{p}{)}\PY{p}{,}
             \PY{l+s+s2}{\PYZdq{}}\PY{l+s+s2}{rho\PYZus{}cu}\PY{l+s+s2}{\PYZdq{}}\PY{p}{:} \PY{p}{(}\PY{l+m+mf}{8.94e3}\PY{p}{,} \PY{l+m+mi}{100}\PY{p}{)}\PY{p}{,}
             \PY{l+s+s2}{\PYZdq{}}\PY{l+s+s2}{rho\PYZus{}fe}\PY{l+s+s2}{\PYZdq{}}\PY{p}{:} \PY{p}{(}\PY{l+m+mf}{7.98e3}\PY{p}{,} \PY{l+m+mi}{100}\PY{p}{)}\PY{p}{\PYZcb{}}

\PY{c+c1}{\PYZsh{} determine mean of function using variable means}
\PY{n}{function\PYZus{}mu} \PY{o}{=} \PY{n}{calculate\PYZus{}weight}\PY{p}{(}\PY{n}{variables}\PY{p}{[}\PY{l+s+s2}{\PYZdq{}}\PY{l+s+s2}{diameter}\PY{l+s+s2}{\PYZdq{}}\PY{p}{]}\PY{p}{[}\PY{l+m+mi}{0}\PY{p}{]}\PY{p}{,} 
                               \PY{n}{variables}\PY{p}{[}\PY{l+s+s2}{\PYZdq{}}\PY{l+s+s2}{length}\PY{l+s+s2}{\PYZdq{}}\PY{p}{]}\PY{p}{[}\PY{l+m+mi}{0}\PY{p}{]}\PY{p}{,} \PY{n}{variables}\PY{p}{[}\PY{l+s+s2}{\PYZdq{}}\PY{l+s+s2}{rho\PYZus{}cu}\PY{l+s+s2}{\PYZdq{}}\PY{p}{]}\PY{p}{[}\PY{l+m+mi}{0}\PY{p}{]}\PY{p}{,} \PY{n}{variables}\PY{p}{[}\PY{l+s+s2}{\PYZdq{}}\PY{l+s+s2}{rho\PYZus{}fe}\PY{l+s+s2}{\PYZdq{}}\PY{p}{]}\PY{p}{[}\PY{l+m+mi}{0}\PY{p}{]}\PY{p}{)}

\PY{n}{variable\PYZus{}sigmas} \PY{o}{=} \PY{p}{[}\PY{n}{variables}\PY{p}{[}\PY{l+s+s2}{\PYZdq{}}\PY{l+s+s2}{diameter}\PY{l+s+s2}{\PYZdq{}}\PY{p}{]}\PY{p}{[}\PY{l+m+mi}{1}\PY{p}{]}\PY{p}{,} \PY{n}{variables}\PY{p}{[}\PY{l+s+s2}{\PYZdq{}}\PY{l+s+s2}{length}\PY{l+s+s2}{\PYZdq{}}\PY{p}{]}\PY{p}{[}\PY{l+m+mi}{1}\PY{p}{]}\PY{p}{,} 
                   \PY{n}{variables}\PY{p}{[}\PY{l+s+s2}{\PYZdq{}}\PY{l+s+s2}{rho\PYZus{}cu}\PY{l+s+s2}{\PYZdq{}}\PY{p}{]}\PY{p}{[}\PY{l+m+mi}{1}\PY{p}{]}\PY{p}{,} \PY{n}{variables}\PY{p}{[}\PY{l+s+s2}{\PYZdq{}}\PY{l+s+s2}{rho\PYZus{}fe}\PY{l+s+s2}{\PYZdq{}}\PY{p}{]}\PY{p}{[}\PY{l+m+mi}{1}\PY{p}{]}\PY{p}{]}

\PY{n}{diameter\PYZus{}output} \PY{o}{=} \PY{n}{first\PYZus{}derivative}\PY{p}{(}\PY{n}{variables}\PY{p}{,} \PY{l+s+s2}{\PYZdq{}}\PY{l+s+s2}{diameter}\PY{l+s+s2}{\PYZdq{}}\PY{p}{)}
\PY{n}{length\PYZus{}output} \PY{o}{=} \PY{n}{first\PYZus{}derivative}\PY{p}{(}\PY{n}{variables}\PY{p}{,} \PY{l+s+s2}{\PYZdq{}}\PY{l+s+s2}{length}\PY{l+s+s2}{\PYZdq{}}\PY{p}{)}
\PY{n}{cu\PYZus{}output} \PY{o}{=} \PY{n}{first\PYZus{}derivative}\PY{p}{(}\PY{n}{variables}\PY{p}{,}\PY{l+s+s2}{\PYZdq{}}\PY{l+s+s2}{rho\PYZus{}cu}\PY{l+s+s2}{\PYZdq{}}\PY{p}{)}
\PY{n}{fe\PYZus{}output} \PY{o}{=} \PY{n}{first\PYZus{}derivative}\PY{p}{(}\PY{n}{variables}\PY{p}{,} \PY{l+s+s2}{\PYZdq{}}\PY{l+s+s2}{rho\PYZus{}fe}\PY{l+s+s2}{\PYZdq{}}\PY{p}{)}

\PY{n}{variable\PYZus{}derivatives} \PY{o}{=} \PY{p}{[}\PY{n}{diameter\PYZus{}output}\PY{p}{,} \PY{n}{length\PYZus{}output}\PY{p}{,} \PY{n}{cu\PYZus{}output}\PY{p}{,} \PY{n}{fe\PYZus{}output}\PY{p}{]}

\PY{n}{sigma} \PY{o}{=} \PY{n}{calculate\PYZus{}sigma}\PY{p}{(}\PY{n}{variable\PYZus{}sigmas}\PY{p}{,} \PY{n}{variable\PYZus{}derivatives}\PY{p}{,} \PY{n}{correlation\PYZus{}matrix\PYZus{}3}\PY{p}{)}

\PY{n}{success\PYZus{}probabilities}\PY{o}{.}\PY{n}{append}\PY{p}{(}\PY{n}{norm}\PY{o}{.}\PY{n}{cdf}\PY{p}{(}\PY{l+m+mi}{22}\PY{p}{,} \PY{n}{function\PYZus{}mu}\PY{p}{,} \PY{n}{sigma}\PY{p}{)}\PY{p}{)}
\end{Verbatim}
\end{tcolorbox}

    \hypertarget{variables-with-correlation.}{%
\section{Variables with
Correlation.}\label{variables-with-correlation.}}

How does the correlation change the solution?

    \begin{tcolorbox}[breakable, size=fbox, boxrule=1pt, pad at break*=1mm,colback=cellbackground, colframe=cellborder]
\prompt{In}{incolor}{29}{\boxspacing}
\begin{Verbatim}[commandchars=\\\{\}]
\PY{n}{correlation\PYZus{}matrix\PYZus{}4} \PY{o}{=} \PY{p}{[}\PY{p}{[}\PY{l+m+mi}{1}\PY{p}{,} \PY{o}{.}\PY{l+m+mi}{2}\PY{p}{,} \PY{o}{.}\PY{l+m+mi}{3}\PY{p}{,} \PY{o}{.}\PY{l+m+mi}{7}\PY{p}{]}\PY{p}{,} \PY{p}{[}\PY{o}{.}\PY{l+m+mi}{2}\PY{p}{,} \PY{l+m+mi}{1}\PY{p}{,} \PY{o}{.}\PY{l+m+mi}{5}\PY{p}{,} \PY{o}{.}\PY{l+m+mi}{6}\PY{p}{]}\PY{p}{,} \PY{p}{[}\PY{o}{.}\PY{l+m+mi}{3}\PY{p}{,} \PY{o}{.}\PY{l+m+mi}{5}\PY{p}{,} \PY{l+m+mi}{1}\PY{p}{,} \PY{o}{.}\PY{l+m+mi}{2}\PY{p}{]}\PY{p}{,} \PY{p}{[}\PY{o}{.}\PY{l+m+mi}{7}\PY{p}{,} \PY{o}{.}\PY{l+m+mi}{6}\PY{p}{,} \PY{o}{.}\PY{l+m+mi}{2}\PY{p}{,} \PY{l+m+mi}{1}\PY{p}{]}\PY{p}{]}

\PY{c+c1}{\PYZsh{}same varibles as previous part}
\PY{n}{sigma} \PY{o}{=} \PY{n}{calculate\PYZus{}sigma}\PY{p}{(}\PY{n}{variable\PYZus{}sigmas}\PY{p}{,} \PY{n}{variable\PYZus{}derivatives}\PY{p}{,} \PY{n}{correlation\PYZus{}matrix\PYZus{}4}\PY{p}{)}

\PY{n}{success\PYZus{}probabilities}\PY{o}{.}\PY{n}{append}\PY{p}{(}\PY{n}{norm}\PY{o}{.}\PY{n}{cdf}\PY{p}{(}\PY{l+m+mi}{22}\PY{p}{,} \PY{n}{function\PYZus{}mu}\PY{p}{,} \PY{n}{sigma}\PY{p}{)}\PY{p}{)}
\end{Verbatim}
\end{tcolorbox}

    Answers

    \begin{tcolorbox}[breakable, size=fbox, boxrule=1pt, pad at break*=1mm,colback=cellbackground, colframe=cellborder]
\prompt{In}{incolor}{30}{\boxspacing}
\begin{Verbatim}[commandchars=\\\{\}]
\PY{k}{for} \PY{n}{i}\PY{p}{,} \PY{n}{answer} \PY{o+ow}{in} \PY{n+nb}{enumerate}\PY{p}{(}\PY{n}{success\PYZus{}probabilities}\PY{p}{)}\PY{p}{:}
    \PY{n+nb}{print}\PY{p}{(}\PY{l+s+s2}{\PYZdq{}}\PY{l+s+s2}{The probability of a motor being less than 22kg with the parameters in \PYZsh{}}\PY{l+s+s2}{\PYZdq{}}\PY{p}{,} \PY{n}{i}\PY{o}{+}\PY{l+m+mi}{1}\PY{p}{,} \PY{l+s+s2}{\PYZdq{}}\PY{l+s+s2}{ is}\PY{l+s+s2}{\PYZdq{}}\PY{p}{,} \PY{n}{answer}\PY{p}{)}

\PY{n+nb}{print}\PY{p}{(}\PY{l+s+s1}{\PYZsq{}}\PY{l+s+s1}{The correlation matrix in \PYZsh{}4 created a}\PY{l+s+s1}{\PYZsq{}}\PY{p}{,} \PY{p}{(}\PY{n}{success\PYZus{}probabilities}\PY{p}{[}\PY{l+m+mi}{2}\PY{p}{]}\PY{o}{\PYZhy{}}\PY{n}{success\PYZus{}probabilities}\PY{p}{[}\PY{l+m+mi}{3}\PY{p}{]}\PY{p}{)}\PY{p}{,}
      \PY{l+s+s2}{\PYZdq{}}\PY{l+s+s2}{ decrease in probability of meeting design parameters.}\PY{l+s+s2}{\PYZdq{}}\PY{p}{)}
\end{Verbatim}
\end{tcolorbox}

    \begin{Verbatim}[commandchars=\\\{\}]
The probability of a motor being less than 22kg with the parameters in \# 1  is
0.98862
The probability of a motor being less than 22kg with the parameters in \# 2  is
0.99015
The probability of a motor being less than 22kg with the parameters in \# 3  is
0.6064653429109432
The probability of a motor being less than 22kg with the parameters in \# 4  is
0.6052891090555454
The correlation matrix in \#4 created a 0.0011762338553977791  decrease in
probability of meeting design parameters.
    \end{Verbatim}

    \begin{tcolorbox}[breakable, size=fbox, boxrule=1pt, pad at break*=1mm,colback=cellbackground, colframe=cellborder]
\prompt{In}{incolor}{ }{\boxspacing}
\begin{Verbatim}[commandchars=\\\{\}]

\end{Verbatim}
\end{tcolorbox}


    % Add a bibliography block to the postdoc
    
    
    
\end{document}
